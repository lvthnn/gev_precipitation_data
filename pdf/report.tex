\documentclass[12pt, a4paper]{article}

\usepackage[scale=0.8]{geometry}

\usepackage[T1]{fontenc}
\usepackage[utf8]{inputenc}

\usepackage[UKenglish]{babel}

\usepackage{amsmath, amsfonts, amssymb}
\usepackage{newtxtext, newtxmath}
\usepackage[scale=0.8]{noto-mono}
\usepackage{bm}

\DeclareMathAlphabet{\mathcal}{OMS}{cmsy}{m}{n}

\usepackage{graphicx}
\usepackage{tabularx}
\usepackage{caption}
\usepackage{cleveref}

\usepackage{booktabs}

\usepackage{lipsum}

\renewenvironment{abstract}
 {\small
  \begin{center}
  \bfseries \abstractname\vspace{-.5em}\vspace{0pt}
  \end{center}
  \list{}{%
    \setlength{\leftmargin}{20mm}
    \setlength{\rightmargin}{\leftmargin}%
  }%
  \item\relax}
 {\endlist}

\captionsetup{
  labelfont = bf,
  labelsep = period,
  font = small
}

\usepackage{minted}

\title{Application of Bayesian latent Gaussian models to
       precipitation data in the context of extreme value theory}
\author{H. Kári Hlynsson}
\date{}

\begin{document}
  
  \maketitle
  
  \begin{abstract}
    This article constructs Bayesian statistical models for describing extremes
    in precipitation using data from South England which ranges over a period
    of 132 years. Three models are considered: (i)
    a null model with fixed parameters, (ii) a model involving a linear time
    trend, and (iii) a Bayesian latent Gaussian model with a piecewise linear
    time trend.\@ Models are fit to data using MCMC sampling and evaluated
    using various goodness-of-fit estimators, such as the WAIC and DIC, and
    empirical quantifiers such as MSE from LOOCV.
  \end{abstract}

  \section{Introduction}
  Extreme value theory is a branch of statistics that deals with observations
  deviating very far from the centre of the underlying distribution they are
  drawn from.\@ The discipline has proven to be very effective in settings 
  where extreme deviations occur with little warning or precedence, such as
  natural disasters, financial crashes, and seemingly superhuman athletic
  achievements. In particular, the generalized extreme value (GEV) distribution
  has seen great use due to the result of the Fisher-Tippett theorem.\@ In
  brief, the theorem states that for a sequence of independent and identically
  distributed random variables \((Y_i)_{i \geq 1}\), the limiting distribution
  of block maxima of the \(Y_i\) is necessarily the GEV distribution.\@
  Although frequentist methods have seen considerable success in employing the
  distribution for various purposes, the Bayesian modelling framework remains
  especially attractive due to its ability to integrate prior knowledge with
  ease, and to capture uncertainty in parameter estimates in settings where
  frequentist derivations of confidence intervals are unknown.

  In this article, three Bayesian statistical models of precipitation extremes
  are constructed. The first model is the most simple; the data are fit to a
  GEV, the parameters of which are fixed in time.\@ The latter two involve a
  time trend, either unbroken or split across intervals, which aim to capture
  the effect of e.g.\@ climate change, which some sources state account for a
  2-5\% increase in precipitation variability globally per year.\@ Another
  source maintains that an increase of 1K in temperature correspond to an
  increase of roughly 7\% in saturation vapor pressure, tightly coupled to
  precipitation intensity, demonstrating the intimate link of global warming to
  increased precipitation due to intensification of the hydrological cycle.\@
  Natural disasters that often accompany periods of intense precipitation, such
  as floods, can cause severe damage to infrastructure and lead to loss of
  life.\@ The complexity of the underlying physical mechanisms and limited
  predictability clearly demonstrate the need for effective statistical models,
  both for preventing and mitigating the damage sustained by such averse
  events, and to motivate policymakers to install protective regulations.\@
  Thus, the application of Bayesian modelling, especially hierarchical methods,
  have the potential to dampen the effects felt from these types of events, and
  to elucidate the underlying mechanisms driving them.
  
  \section{Exploration of data}
  The data consist of \(n = 48212\) observations of cumulative precipitation
  in millimetres over a 24-hour period from a catchment in South England. 
  These data are aggregated into yearly maxima, shown as a time series in
  Figure \ref{fig:timeseries_precip}. Henceforth, precipitation or cumulative
  precipitation will be understood as being recorded over a 24-hour period. 

  \begin{table}
    \centering
    \caption{Table of summary statistics for the South England precipitation
    data, grouped by decade. The second column, \(s_y\), denotes the sample
    standard deviation. 95\% confidence intervals are computed for the mean 
    using the standard \(t\)-distribution formula.}
    \begin{tabularx}{\textwidth}{XXXXX}
      \toprule
      Decade & No. obs. & \(\overline y\)  & \(s_y\) & 95\% CI      \\
      \midrule
      1890   &  9       & 22.3             & 5.10    & [18.4, 26.2] \\
      1900   & 10       & 23.3             & 7.64    & [17.8, 28.8] \\
      1910   & 10       & 34.3             & 6.94    & [29.4, 39.3] \\
      1920   & 10       & 30.3             & 12.2    & [21.7, 39.0] \\
      1930   & 10       & 26.0             & 11.1    & [18.0, 34.0] \\
      1940   & 10       & 28.8             & 5.09    & [25.1, 32.4] \\
      1950   & 10       & 31.6             & 12.5    & [22.7, 40.5] \\
      1960   & 10       & 30.3             & 9.88    & [23.2, 37.4] \\
      1970   & 10       & 28.9             & 7.22    & [23.8, 34.1] \\
      1980   & 10       & 34.0             & 7.98    & [28.3, 39.7] \\
      1990   & 10       & 30.9             & 7.91    & [25.3, 36.6] \\
      2000   & 10       & 37.8             & 13.0    & [28.5, 47.0] \\
      2010   & 10       & 30.7             & 7.89    & [25.1, 36.4] \\
      2020   &  3       & 24.5             & 4.58    & [13.1, 35.9] \\
      \bottomrule
    \end{tabularx}
    \label{tab:precip_summary}
  \end{table}

  \begin{table}
    \centering
    \caption{Five number summary of the columns in the South England data.}
    \begin{tabularx}{\textwidth}{XXX}
      \toprule
              & Year & Max. precip. [mm] \\
      \midrule
      Min.    & 1891 & 10.35             \\
      1st Qu. & 1924 & 22.82             \\
      Median  & 1956 & 27.95             \\
      Mean    & 1956 & 29.88             \\
      3rd Qu. & 1989 & 36.12             \\
      Max.    & 2022 & 64.22             \\
      \bottomrule
    \end{tabularx}
  \end{table}
  
  \begin{figure}[H]
    \begin{center}
      \includegraphics[width=\textwidth]{decade.pdf}
    \end{center}
    \caption{(\textbf{A}) Time-series plot of annual maxima of precipitation.
    \textbf{(B1)} Plot of mean maximum precipitation per year by decade.\@ The
    blue line represents a least squares line and the shaded region represents
    the 95\% confidence interval.\@ \textbf{(B2)} Residual plot of mean maximum
    precipitation per year by decade.}
    \label{fig:timeseries_precip}
  \end{figure}

  \section{Bayesian modelling framework}
  Throughout the course of this article we will explore three different
  modelling approaches, all employing a Bayesian framework and MCMC sampling
  for posterior inference. The notation \(\mathcal M_j\) will be used to refer
  to the models.

  The first model we consider is a naive fixed-parameter model of the form
  \begin{align*}
    Y_t &\sim \mathrm{GEV}(\mu, \sigma, \xi), \\
    \mu &\sim \mathcal N(0, \tau_\mu^2), \\
    \sigma &\sim \mathrm{Exp}(\lambda_\sigma), \\
    \beta &\sim \mathrm{Beta}(4, 4)
  \end{align*}
  where \(Y_t\) denotes the maximum annual precipitation at time \(t\).

  The second model that is posed is a time-variant model with a linear trend
  that can be written in the form 
  \begin{align*}
    Y_t &\sim \mathrm{GEV}(\mu_t, \sigma, \xi),  \\
    \mu_t &:= \mu_0 \{1 + \Delta (t - t_c)\}, \\
    \mu_0 &\sim \mathcal N(0, \tau_\mu^2), \\
    \Delta &\sim \mathcal N(0, \tau_\Delta^2), \\
    \sigma &\sim \mathrm{Exp}(\lambda_\sigma), \\
    \xi &\sim \mathrm{Beta}(4, 4)
  \end{align*}
  where \(t_c\) represents the median time of the data set, calculated as
  \(t_c = \lfloor \frac{t_N - t_1}{2} \rfloor\). The last model that we explore 
  is a hierarchical Bayesian model of the form
  \begin{align*}
    Y_t &\sim \mathrm{GEV}(\mu_t, \sigma, \xi), \\
    \mu_t &:= \sum_{k = 1}^K \beta_k (t - t_0) \mathbf 1_{t > t_k}, \\
    \sigma &\sim \mathrm{Exp}(\lambda_\sigma), \\
    \xi &\sim \mathrm{Beta}(4, 4).
  \end{align*}
  In this model, the random effects vector \(\bm \beta\) is assigned three
  different priors;
  \begin{align*}
    \bm \beta \sim \mathcal N(\bm 0, \Sigma_\beta),
    \quad \Sigma_\beta = \mathrm{diag}(\sigma_\beta), \quad
    \sigma_\beta \sim \mathrm{Exp}(\lambda)
  \end{align*}
  In other words,
  \[
    \mu_t := \gamma_t (t - t_0) \quad \text{where} \quad
    \gamma_t := \sum_{}
  \]
  Another prior that we will use is the Gaussian random walk. Let
\end{document}
