\documentclass[12pt, a4paper]{article}

\usepackage[scale=0.8]{geometry}
\usepackage[T1]{fontenc}
\usepackage[utf8]{inputenc}
\usepackage{amsmath, amsfonts}
\usepackage{newtxtext, newtxmath}
\usepackage[scale=0.8]{noto-mono}
\usepackage{bm}

\usepackage{minted}

\title{Application of Bayesian latent Gaussian models to
       precipitation extremes in the context of extreme value theory}
\author{Kári Hlynsson}
\date{July 2024}

\begin{document}
  
  \maketitle
  
  \begin{abstract}
    \noindent
    Effective modelling of extreme precipitation events have the potential
    to reduce casualty and infrastructural damage as a result of natural
    disasters such as those caused by floods. The application of Bayesian
    hierarchical models (BHMs), particularly Bayesian latent Gaussian models
    (BLGMs), has seen substantial success in mitigating the effects of these
    events and motivating preventative measures. 
    In this article, BLGMs are applied to precipitation data where the response
    layer is modelled using the generalized extreme value (GEV) distribution.
    Three models are compared, namely (i) a baseline model, (ii) a time-trend
    model, and (iii) a piecewise-linear time trend model.\@ Important concepts
    from extreme value theory are reviewed to give the reader perspective.
  \end{abstract}

  \section{Introduction}
  \subsection{Motivation and statement of purpose}
  Disastrous and often unlikely events such as floods, heatwaves or financial
  market crashes can occur suddenly and without warning, leading to
  loss of life and damage to infrastructure.\@ Accurate modelling of such
  events has been the subject of great interest in the statistical community,
  and has recently seen substantial success when considered in the context of
  extreme value theory. In particular, Bayesian hierarchical models (BHMs) have
  proven to be especially desirable due to their ability to incorporate prior
  knowledge and to obtain a measure of uncertainty, for example by using the 
  posterior predictive density when predicting likely future outcomes.

  In this article, models of maximum cumulative precipitation in the period of
  24 hours by year is modelled using the generalized extreme value (GEV)
  distribution, which is a natural distribution to block maxima data (such as
  yearly maxima), as follows from the Fisher-Tippett-Gnedenko theorem, briefly
  discussed in the next section. The distribution has three parameters,
  \(\mu\), \(\sigma\), and \(\xi\), which are often referred to as the 
  \textit{location}, \textit{scale}, and \textit{shape} parameters,
  respectively. Three models are considered, with increasing complexity. The
  first model, the null model, claims that the precipitation data are
  modelled with no trend, ignoring climate effects. The second model uses a
  time trend in the location parameter of the GEV distribution. Lastly, the 
  third model uses piecewise linear trends to predict changes in location
  parameter with time. Comparison of the aforementioned models has the
  potential to eludicate effects of climate change on extreme precipitation
  events.

  \section{Extreme value theory}
  Fisher-Tippett theorem states that [...]. The GEV has likelihood
  \[
    \mathcal L(y \mid \mu, \sigma, \xi) = \sigma^{-1} \eta(y)^{\xi + 1}
                                          \exp\big(-\eta(y)\big),
  \]
  where
  \[
    \eta(y) = \begin{cases}
      \left[1 + \xi \left(\frac{y - \mu}{\sigma}\right)\right]^{-1/\xi}
      &\text{if } \xi \neq 0, \\
      \exp\left(\frac{y - \mu}{\sigma}\right) &\text{if } \xi = 0.
    \end{cases}
  \]
  Thus, the likelihood of the observed data is 
  \begin{align*}
    \mathcal L(\bm y \mid \mu, \sigma, \xi)
    &= \prod_{i = 1}^n \mathcal L(y_i \mid \mu, \sigma, \xi) \\
    &= \sigma^{-n} \prod_{i = 1}^n \eta(y_i)^{\xi + 1} \exp\big(
       -\eta(y_i)\big),
  \end{align*}
  and by taking the log of the above we arrive at the log likelihood:
  \[
    \ell(\bm y \mid \mu, \sigma, \xi)
    = -n \log(\sigma) + \sum_{i = 1}^n (1 + \xi) \log \eta(y_i)
                      - \sum_{i = 1}^n \eta(y_i).
  \]


  
\end{document}
